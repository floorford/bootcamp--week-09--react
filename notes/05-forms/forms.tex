\section{Forms}

One thing that we have to be particularly careful with is form inputs.
\\

A component's state is meant to represent \textit{any} values that change in the DOM. That means that, as the user types into a form input, the state should be updated.

\inputminted{jsx}{05-forms/figures/01-input.jsx}


\pagebreak


So, we should store the current value of the input item in the \texttt{state}:

\inputminted{jsx}{05-forms/figures/02-input-state.jsx}

The input's value is now stored in the state but \textit{we can't change the value of the input!}
\\

So, we'll need to update the state when the value changes. We can use the \textbf{event object} to do this. React passes the same event object that we got given when we added events to the DOM.
\\

We can use the \texttt{currentTarget} property of the event object to get back whatever DOM element our event was called on, in this case the \texttt{<input>}. We can then get the value of the \texttt{<input>} using the DOM \texttt{value} property:

\begin{minted}{js}
    // e is the standard DOM event object
    handleChange(e) {
        // e.currentTarget: DOM element we attached the event handler to
        // use the value property to read its current value
        this.setState({ input: e.currentTarget.value });
    }
\end{minted}

We'll need to add an event handler to the \texttt{<input>}:

\begin{minted}{jsx}
    <input
      onChange={ this.handleChange }
      value={ this.state.input }
      name={ this.props.name }
      className="form-control"
    />
\end{minted}

And don't forget to \texttt{bind} the \texttt{handleChange} method:

\begin{minted}{jsx}
constructor(props) {
    // ...
    this.handleChange = this.handleChange.bind(this);
}
\end{minted}

Now when a user types into the input the state will update too. This is called a \textbf{controlled component}. All form elements in a React app should be controlled components.
\\

Some libraries/frameworks like Vue.js support ``two-way data binding'' which removes the need to handle form input manually. It seems like a good idea initially, but actually once you start building more complex apps it stops being particularly useful and can lead to all sorts of hard-to-spot bugs. You can remove a lot of the repetitive boilerplate just by using sub-components carefully.


\pagebreak


\section{Notes About React Form Elements}

There are a few things to be aware of when using form elements with React:

\begin{itemize}
    \item If you set the \texttt{value} attribute you \textit{have} to set an \texttt{onChange} event handler (otherwise it would be impossible for the user to change the input's value)
    \item The \texttt{onChange} event is closer to the \texttt{input} DOM event: it gets fired on every keystroke for an input.
    \item If you want to set a default value you should use the \texttt{defaultValue} attribute.
    \item The \texttt{value} property works for all form elements, including ones like \texttt{<select>} which don't have a \texttt{value} property when using native DOM.
\end{itemize}


\section{Additional Resources}

\begin{itemize}[leftmargin=*]
    \item \href{https://reactjs.org/docs/forms.html}{React: Forms}
\end{itemize}
