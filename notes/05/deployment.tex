If you've used \texttt{npm init react-app}, you can simply run \texttt{npm build} to create a copy of your site that you can run on any server. It will create a directory called \texttt{build} that contains an \texttt{index.html} file and the fully compressed and transpiled JS.
\\

You'll need to rerun \texttt{npm build} every time you want to publish a new version of the site - it doesn't keep itself up to date.
\\

You shouldn't add the \texttt{build} directory to Git, as it's easy to recreate by just running \texttt{npm build}.


\section{Deploying to GitHub Pages}

As a static site you can deploy a React app to GitHub Pages (GitHub hosts your site from your repo).

\begin{itemize}
    \item Install the package needed: \texttt{npm install gh-pages}
    \item Add \texttt{"homepage": "https://<github username>.github.io/<repo name>/} to your \texttt{package.json} file.
    \item Add \texttt{"predeploy": "npm run build"} to your \texttt{package.json} in the \texttt{"scripts"} object.
    \item Add \texttt{"deploy": "gh-pages -d build"} to your \texttt{package.json} in the \texttt{"scripts"} object.
    \item Then, when ready to deploy to GitHub Pages: \texttt{npm run deploy}
    \item Wait. The first deployment will be fast, but future deployments can take up to 10 minutes.
    \item Visit \texttt{https://<github username>.github.io/<repo name>/} to view your React app
\end{itemize}
