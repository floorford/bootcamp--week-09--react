You can deploy you React site to GitHub pages easily.
\\

We'll use an NPM package called \texttt{gh-pages} to make our lives easier.

\section{Setup}

First, install the \texttt{gh-pages} package

\begin{minted}{bash}
    npm install gh-pages
\end{minted}

Next, update the \texttt{homepage} property in your \texttt{package.json}:

\begin{minted}{json}
    "homepage": "https://<username>.github.io/<repo>/",
\end{minted}

Then add the following two lines to the \texttt{scripts} object in your \texttt{package.json}:

\begin{minted}{json}
    "predeploy": "npm run build",
    "deploy": "gh-pages -d build"
\end{minted}


\pagebreak


\section{Deploying}

Whenever you want to update your site run:

\begin{minted}{bash}
    npm run deploy
\end{minted}

If your forget to do this, it won't update!
\\

Deployment can take a minute or two sometimes. You can go to \\ \texttt{https://github.com/<github-username>/<repo>/deployments} to see if a deployment is in progress.
\\

Once deployment is done, you can visit:\\ \texttt{https://<github-username>.github.io/<repo>/}


\section{React Router}

GitHub pages doesn't support single-page web-app URL routing, where all URLs get redirected to a single file. So we'll need to update our React app to use the \texttt{<HashRouter>} component instead of the \texttt{<BrowserRouter>} component:

\begin{minted}{js}
    import {
      HashRouter as Router,
      // ... other imports
    } from "react-router-dom";
\end{minted}

\pagebreak


\begin{infobox}{Hash Routing}
    ``Hash routing'' means that rather than using a standard sub-directory style URL like \texttt{https://github.com/spoons/blah} it adds a \texttt{\#} before it: \\ \texttt{https://github.com/spoons/\#/blah}.
    \\

    This looks very similar to a user, but means something completely different to a web-browser. The \texttt{\#} looks for an \texttt{id} on the current page, meaning that all requests get sent to the same page. ReactRouter can then intercept this value and make it work as before.
    \\

    There are \href{https://www.quora.com/Are-hashbang-URLs-a-recommended-practice}{various issues} associated with using hash URLs, but if you don't have control of the server setup then it's often the only option.
\end{infobox}


\section{Additional Resources}

\begin{itemize}[leftmargin=*]
    \item \href{https://pages.github.com}{GitHub Pages}
    \item \href{https://github.com/tschaub/gh-pages}{\texttt{gh-pages} Package}
\end{itemize}
